
\documentclass{article}
\usepackage[euler-digits]{eulervm}
\usepackage{lipsum}
\usepackage{mathtools,amsthm,amssymb}
\usepackage[normalem]{ulem}
% this is a comment, all comments start with %
\begin{document}
Heoolo world!

\begin{itemize}
\item The first item
\item The second etc 
\end{itemize}

Binomial formula:
There are $\binom{2n+1}{n}$ with $1/(2n+1)$ of
these have all partial sums positive

Greek letters:
Alpha symbol is $\alpha$
Beta symbol is $\beta$
Gamma symbol is $\gamma$


%Phi : \Phi, \phi and \varphi
%Chi : \Chi and \chi
%Omega: \Omega and \omega


Lambda symbol is $\lambda$
Delta symbol is $\delta$
%Eta: \Eta and \eta
%TheTa: \Theta and \theta
%Kappa: \Kappa and \kappa


Epsilon symbol is $\epsilon$
%Mu: \Mu and \mu


% underscore for subscript and caret for super script
Powers and indices:
$k_{n+1} = n^2 + k_n^2 - k_{n-1}$

% for superscript or subscript with more than one digit surround with {}
$n^{221}$

$n^{p+1}$

$p_{k+1}$

% fractions
Fractions: 
$\frac{n!}{k!(n-k)!} = \binom{n}{k}$

% frac syntax = \frac{numerator}{denominator}


% roots

Square root of a fraction:

$\sqrt[4]{\frac{a}{b}}$

Nth root of something
$\sqrt[n]{1+x+x^2+x^3+\dots+x^n}$

% sum and product symbols with \sum and \prod

$\sum_{i=1}^{i=10} t_i$


Factorial formula:

$n! = 1.2.3....n = \prod_{k=1}^{n}k, \\ integer  n\geq0.$

% 2d parenthesized matrix with vertical horizontal and diagonal dots

Matrices:

$
A_{m,n} = \begin{pmatrix} a_{1,1} & a_{1,2} \\
  a_{2,1} & a_{2,2}
  \end{pmatrix}
  $

  % bmatrix gives square bracket matrix
$
B_{m,n} = \begin{bmatrix} b_{11} & b_{12} \\
  b_{21} & b_{22}
  \end{bmatrix}
$


Set and logic symbols in latex:

Set notation is $\left\{ {x,y,z} \right\}$

Empty set is $\varnothing$ or $\emptyset$

Set intersection is $\cap$

Set union is $\cup$

Set difference is $\setminus$

Cartesian product is $\times$

Set membership given by $\in$

Universal Quantifier is $\forall$

Existential Quantifier is $\exists$

Cardinality of a set is $\left\vert{S}\right\vert$

Subset is $\subseteq$

Proper subset is $\subset$

SuperSet is $\supseteq$

Proper superset is $\supset$

Negation of anything is start with not $\not \in$

Mapping from A to B is $f: A \to B$

If f is injective, it is $f: A \rightarrowtail B$

If f is surjective, it is $f: A \twoheadrightarrow B$

if f is a bijection it is $f: A \leftrightarrow B$

Failing powers of factorial is $x^{\uline{m}}$

Formula for exponent of a prime p in (n!)'s unique factorization:

\DeclarePairedDelimiter{\ceil}{\lceil}{\rceil}
\DeclarePairedDelimiter{\floor}{\lfloor}{\rfloor}

% floor* will have vertical resizable floor delemiters
$\epsilon_p(n!) = \floor*{\dfrac{n}{p}} + \floor*{\dfrac{n}{p^2}} +\floor*{\dfrac{n}{p^3}}
+ \hdots = \sum\limits_{k\geq1} \floor*{\dfrac{n}{p^k}} $


\newcommand{\fallingfactorial}[1]{%
  ^{\underline{#1}}%
}


Failing powers of factorial is $x^{\uline{m}}$
\[
x\fallingfactorial{n}={\overbrace{x(x-1)\dots(x-n+1)}^{\text{$n$ factors}}}
\]


\end{document}