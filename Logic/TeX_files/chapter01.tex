\chapter{Common concepts}




\section{Propositions}

\begin{definition}[Proposition]
	A proposition is a statement that is either true or false (and usuallly has a variable).
\end{definition}

\begin{definition}[Compound Proposition]
	Compound propositions: Combine multiple propositions using and/or/not/implies
\end{definition}


\begin{definition}[Predicate]
	A predicate is a proposition whose truth value dedpends on one or more variables.
\end{definition}

For e.g.
	"n is a perfect square"
	P(n) ::= "n is a perfect square"
	P(4) is true, P(5) is false
	
\begin{definition}[Quantifier]
A quantifier is an operator that specifies how many individuals in the domain of discourse satisfy an open formula. ($\forall$ or $\exists$)
\end{definition}

For instance, the universal quantifier $\forall$  in the first order formula ${ \forall xP(x)}$ expresses that everything in the domain satisfies the property denoted by $P$. On the other hand, the existential quantifier ${ \exists }$ in the formula ${ \exists xP(x)}$ expresses that there is something in the domain which satisfies that property

\begin{lemma}[Negating Quantifiers]
	Universal negation: 
	\begin{center}
	$ \neg (\forall x \in D, P(x)) \equiv \exists x \in D, \neg P(x). $
	\end{center}

	Existential negation:
	\begin{center}
	$ \neg{(\exists x \in X, P(x))} \equiv \forall x \in D, \neg P(x). $
	\end{center}
\end{lemma}

\begin{lemma}[Order of mixed quantifiers]
	When multiple quantifiers are of the same type, e.g. all $\forall$ or all $\exists$, then it is ok.
	\begin{center}
		$ \forall x \forall y . Likes(x,y) \equiv \forall y \forall x . Likes(x,y) $
	\end{center}
	
	
	But if quantifiers are of different type,
	then they do not mean the same
	\begin{center}
		$\forall x \exists y . Likes(x,y) \ne \exists y \forall x .Likes(x, y)$
	\end{center}

\end{lemma}

\begin{definition}[Validity]
	A propositional formula is called valid when it evaluates to $T$ no matter what truth
	values are assigned to the individual propositional variables.
\end{definition}

\begin{definition}[Satisfiability]
A proposition is satisfiable if some setting of the variables makes the proposition
true
\end{definition}


E.g.
e, P $\land$ $\neg$Q is satisfiable because the expression is true if P is true
or Q is false.

\section{Induction}

\begin{definition}[Well Ordering principle]
	Every nonempty set of nonnegative integers has a smallest element.	
\end{definition}


\begin{definition}[Induction]
	Let P(n) be a predicate. 
	If
	\begin{itemize}
		\item P (0) is true, and
		\item P (n) IMPLIES P(n + 1) for all nonnegative integers, n,  
	\end{itemize}
	then
	\begin{itemize}
		\item P (m) is true for all nonnegative integers, m.
	\end{itemize}

In other words:
\infer{\forall m \in \mathbb{N}. P(m)}{%
	P(0)
	& \forall n \in \mathbb{N}. P(n) \implies P(n+1)
}
	
\end{definition}

\begin{definition}[Invariant]
	A property that is preserved through a series of operations or steps is known as an
	invariant
\end{definition}


\begin{lemma}[Invariant method]
	If you would like to prove that some property NICE holds for every
	step of a process, then it is often helpful to use the following method:
	\begin{itemize}
		\item Define $P(t)$ to be the predicate that NICE holds immediately after step t.
		\item Show that $P(0)$ is true, namely that NICE holds for the start state.	
		\item Show that
		$\forall t \in \mathbb{N}. P(t) \implies P(t + 1)$;
		namely, that for any $t \ge 0$, if NICE holds immediately after step t, it must
		also hold after the following s
	\end{itemize}
\end{lemma}

\begin{definition}[Strong Induction]
	Let P(n) be a predicate
	\begin{itemize}
		\item $P(0)$ is true and 
		\item $P(1), P(2), ... P(n)$ together imply $P(n+1)$ 
	\end{itemize}
	then $P(n)$ is true for all $n \in \mathbb{N}$
\end{definition}